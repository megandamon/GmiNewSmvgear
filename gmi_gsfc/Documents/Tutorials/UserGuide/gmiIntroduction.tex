
\pagenumbering{arabic}			% Arabic page numbers from now on
%
\chapter[Introduction]{Introduction} \label{chap:introduction}
%\section{Overview of the Model}
%\pagenumbering{arabic}			% Arabic page numbers from now on
%
The Global Modeling Initiative (GMI) 
\footnote{http://gmi.gsfc.nasa.gov/gmi.html}
was initiated under the auspices of the Atmospheric Effects of Aircraft 
Program (AEAP) in 1995. 
The goal of GMI is to develop and maintain a state-of-the-art
modular 3-D chemistry transport model (CTM) that can be used
to assess the impact of various natural and anthropogenic
perturbations on atmospheric composition and chemistry, including,
but not exclusively, the effect of aircraft.

The Atmospheric Chemistry Modeling and Analysis Program (ACMAP)
 has selected the approach of GMI to serve
both as an assessment facility and a testbed for model improvements
for future assessment in all areas of atmospheric chemistry.
%
The goals in the design of GMI as an assessment tool are \cite{Kinnison-etal01}
%
\begin{enumerate}
\item The model should be well-characterized and thoroughly tested
      against observations.
\item The model should be able to test and compare a diversity of
      approaches to specific processes by being able to easily swap
      modules containing different formulations of chemical
      processes, within a common framework.
\item The model should be optimized for computational efficiency and
      be able to run on different platforms.
\item Model results should be examined by a large representation of
      the scientific community, thus faciliting consensus on the significance
      of assessment results.
\item Ultimately, the model integration could provide a unique assessment
      capability for other anthropogenic impacts of concern by providing
      a testbed for other algorithms and intercomparisons used in
      assessment of those issues.
\end{enumerate}
%
Many elements of the GMI model address these goals.
The GMI model is a modular chemistry-transport model (CTM) with
the ability to carry out multi-year assessment simulations as
well as incorporate different modules, such as meteorological
fields, chemical mechanisms, numerical methods, and other modules
representing the different approaches of current models. This
capability facilitates the understanding of the differences
and uncertainties of model results.

The testing of GMI results against observations is a high priority
of GMI activities. Science Team members contribute by either
supplying a particular module and/or contributing to the analysis
of the results and comparison with atmospheric observations
\cite{Douglass-etal99, Rotman-etal01, Strahan-Duncan-Hoor07, Meskhidze-etal07}. 
Application of the model to the potential impacts of
stratospheric aircraft emissions is presented in \cite{Kinnison-etal01}. 
The model has been employed to investigate the effects of stratospheric aircraft
emissions on polar stratospheric clouds \cite{Considine-etal00} and simulate
ozone recovery over a 36-year time period \cite{Considine-etal04}.

Besides acting as a testbed for different modules, GMI will also
act as a 3-D assessment facility. The GMI modular code is currently
implemented at NASA/Goddard Space Flight Center (the core institution).
The core institution is responsible for: 
%
\begin{itemize}
\item Integrating and testing components of the GMI model, 
\item Making the code readable, flexible and easy to maintain
\item Maintaining coding standards which will make the model 
      portable to different platforms
\item Carrying out assessment calculations, and 
\item Providing first-order results and diagnostics for 
      analysis by team members. 
\end{itemize}
%
The current  version of the code has been developed to run on a variety of 
computing platforms, both with single and multiple processors 
(Linux clusters, SGI Origin series, HP Compaq SC45, 
single processor workstations, etc.).

This report is intended to familiarize users with the GMI code.
Users will be able to 
\begin{itemize}
\item Have information on the code structure (Chapter \ref{chap:structure}).
\item Obtain instructions on how to obtain the code, install it, 
      compile it, run it on any platform (Chapter \ref{chap:installation}).
\item Have knowledge of all the input and output files involved in the
      code (Chapter \ref{chap:files}, Appendix \ref{chap:netcdf} and
      Appendix \ref{chap:rcFile}).
\item Carry out specific and restart runs (Chapter \ref{chap:runs}).
\item Learn how to make changes in the code (Chapter \ref{chap:changes}).
\item Execute useful script tools needed, for instance, to search
      for words, to produce restart input resource files, etc.
      (Chapter \ref{chap:script}).
\item Learn basic CVS commands (Chapter \ref{chap:cvs}).
\item Analyze the parallel performance of the code 
      (Chapter \ref{chap:performance}).
\item Be familiar with include files used to select the desired architecture,
      to set up compilation options, etc. (Appendix \ref{chap:inc_files}).
\item Know how to carry out a single or multiple processor run 
      (Appendix \ref{chap:proc_runs}).
\item Use resource file features (Appendix \ref{chap:features}).
\item Know the species used in the code (Appendix \ref{chap:listSpecies}).
\end{itemize}

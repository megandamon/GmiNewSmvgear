
\chapter[Script Tools]{Script Tools} \label{chap:script}

\section{Internal Scripts} \label{section:internal}
The GMI code comes with several scripts that allow users to perform 
commands such as searching from words within the code, counting the
number of lines in the source code, constructing a restart input
namelist file, etc.
%
All the scripts are located in the directory
{\em Shared/GmiScripts} and can be executed from
anywhere in the code. They are:
%
\begin{description}
\item[check\_ver] searches for all file names in the current directory
   containing "search\_string" and replaces the first instance of
   "search\_string" with "replace\_string". \\
   Usage:  \mbox{ chname  search\_string  replace\_string}
\item[clgem] deletes a number of the files created when {\em gmi} is run.
\item[doflint] runs the flint Fortran source code analyzer on the GMI code. \\
   It can be run on any machine where flint is available (e.g., tckk), and
   where the GMI code has been installed and compiled. \\
   Usage: \mbox{ doflint }
\item[gmi\_fcheck] can be used to run the Flint source code analyzer on
   gmi\_gsfc.
\item[gmi\_fcheckrm] can be used after "gmi\_fcheck" is run on the gmi\_gsfc
   code to strip out Flint messages that are of no consequence.  
   Gmi\_fcheckrm uses gmi\_fcheck.out as its input file, and produces a new 
   file called gmi\_fcheckrm.out.  
\item[grabf] grabs all of the ".f90" files and puts them in $\$$gmi/CODE.
   These files then can be ftped to a machine with access to the FORTRAN
   Lint (flint) source code analyzer tool.
\item[lastmod] lists all files in reverse order of 
   when they were last modified. It is useful for determining which 
   routines a user has modified since they were last installed in the code. \\
   Usage:  lastmod [tail\_num]
\item[lastmod\_all] lists all files in reverse order of when 
   they were last modified. It is useful for determining which routines 
   a user has modified since they were last installed in the code. \\
   Usage:  lastmod\_all [tail\_num]
\item[line\_count\_gmi] does a variety of line counts on the GMI source code.
\item[list\_species] lists all the species labels for the selected chemical
    mechanism (environment variable CHEMCASE). \\
   Usage: list\_species
\item[lnsdat] (lns (symbolic link) data) symbolically links the GMI input 
   file directory to "gmi\_data" in the current directory.  
   The GMI namelist file can then point to a generic gmi\_data directory.
\item[mk\_rstnl] (make restart namelist) constructs a restart input namelist 
   file (see Section \ref{sec:Restart} for more information).
\item[savit] creates a clean copy of a GMI code tree (considered only
   files with extension [.F90|.c|.h]) in a tmp
   directory, tars it up into a tarfile named gmisav.tar, and puts this
   file in the directory where your gmi directory resides.  The tmp
   files are then deleted.
\item[savset] creates a clean copy of the GMI setkin files
   (considered only files with extension [.F90|.c|.h]) in a tmp
   directory, tars it up into a tarfile named setsav.tar, and puts this
   file in the directory where your gmi directory resides.  The tmp
   files are then deleted.
\item[seabf] searches through the ".F90" source files for a
   particular string (case insensitive). Output comes to the screen
   unless an optional second file\_name argument is provided. \\
   Usage: seabf search\_string [file\_name]
\item[seabf\_word] searches through the ".F90" source files for a
   particular word (case insensitive).  Output comes to the screen
   unless an optional second file\_name argument is provided. \\
   Usage:  seabf\_word search\_string [file\_name]
\item[seac] searches through the .c source files for a
   particular string (case insensitive).  Output comes to the screen
   unless an optional second file\_name argument is provided. \\
   Usage:  seac search\_string [file\_name]
\item[seah] searches through the ".h" include files for a
   particular string (case insensitive). Output comes to the screen unless
   an optional second file\_name argument is provided. \\
   Usage:  seah search\_string [file\_name]
\item[seah\_word]  searches through the ".h" include files for a
   particular word (case insensitive).  Output comes to the screen unless
   an optional second file\_name argument is provided. \\
   Usage:  seah\_word search\_string [file\_name]
\item[sealf] searches through the ".f90" source files for a
   particular string (case insensitive).  Output comes to the screen
   unless an optional second file\_name argument is provided. \\
   Usage:  sealf search\_string [file\_name]
\item[seamf] searches through the "Makefile" and  "Makefile.cpp" source files
   for a particular string (case insensitive).  Output comes to the screen
   unless an optional second file\_name argument is provided. \\
   Usage:  seamf search\_string [file\_name]
\end{description}


\section{Production Tools} \label{section:production}
There is a collection of python scripts for doing GMI production runs. 
These scripts create namelist files for each segment of the production run 
(usually date segments), start each segment of a production run,
monitor the runs, and sends status updates to a designated email address.
The following scripts can be used seperately or together for doing productions
runs:
\begin{description}
\item[MakeNameLists.py] creates namelist files by using a base namelist \\ 
\item[ChainRuns.py] runs multiple segments of a GMI production run, monitors 
   the progress, and sends status updates
\end{description}

The first step in using either of these scripts is to edit the 
{\em ExperimentConstants.py} file.  The following are the configurable options
inside this file:
\begin{description}
\item[MAILTO] a single email address or a comma seperated list of email addresses to send updates to \\
\item[NUM\_CPUS] total number of processors (slaves+master) \\
\item[BASE\_NAMELIST] the base namelist file \\
\item[NUM\_NAMELISTS] the number of total namelists to be create. We suggest setting this to 12 \\
\item[START\_MONTH] usually "jan" for january; all other months follow the same three letter abbreviation \\
\item[NUM\_ARGS] only developers should change this option
\item[NAMELISTS] the name of the file the will contain a list of all the namelists in the production run; 
  they are listed in the order of execution \\
\item[RUNREF] the base batch file for submission to the discover PBS batch system \\
\item[QUEUEMINUTESALLOWED] the number of minutes a segment should take to run, 
   including the estimated wait time in the queue
\end{description}

Next, a run directory should be created.  The run directory should have a sample RUNREF file and a
BASE\_NAMELIST in it; in this example we use the files {\em run\_ref} and {\em sample.in}, respectively.

In the {\em run\_ref} file, make sure you change the {\em group\_list}, the {\em workDir} (which is the run directory), 
and the path to the executable (GEMHOME).  Also, check the number of cpus.

For the BASE\_NAMELIST file, check that the restart file is correct, and other input file names (anything that contains a string {\em infile\_name}).
This file will be used as a base for all your production segments. The job name, date, restart file (previous month), and the number of days to run will be
changed by the script. 

\begin{remark}
If you limit the number of segments to 12 (one year) you should not have to make any other
modifications EXCEPT when you are intending to run a February segment with 29 days. You will have to change the namelist variable (after you run
the {\em MakeNameLists.py} script) to tfinal\_days to 29.  Users are urged to use caution when using the {\em MakeNameLists.py} and check their work carefully.
\end{remark}

Once you've prepared the run directory execute the {\em MakeNameLists.py} script from the run directory (substitute your own directory names here):
\begin{verbatim}
        % cd runDirectory
        % python /yourpath/MakeNameLists.py -d `pwd`
\end{verbatim}

You should see output similiar to this:

\begin{verbatim}
        nameListPath = /yourpath
        nameListFile = sample.in
        numberOfNameLists = 12
        startMonth = jan
        startYear = 1994
        start month 0 jan
        gmit_sample_1994_feb feb1994.list 28 940201 940228 gmit_sample_1994_jan
        gmit_sample_1994_mar mar1994.list 31 940301 940331 gmit_sample_1994_feb
        gmit_sample_1994_apr apr1994.list 30 940401 940430 gmit_sample_1994_mar
        gmit_sample_1994_may may1994.list 31 940501 940531 gmit_sample_1994_apr
        gmit_sample_1994_jun jun1994.list 30 940601 940630 gmit_sample_1994_may
        gmit_sample_1994_jul jul1994.list 31 940701 940731 gmit_sample_1994_jun
        gmit_sample_1994_aug aug1994.list 31 940801 940831 gmit_sample_1994_jul
        gmit_sample_1994_sep sep1994.list 30 940901 940930 gmit_sample_1994_aug
        gmit_sample_1994_oct oct1994.list 31 941001 941031 gmit_sample_1994_sep
        gmit_sample_1994_nov nov1994.list 30 941101 941130 gmit_sample_1994_oct
        gmit_sample_1994_dec dec1994.list 31 941201 941231 gmit_sample_1994_nov
\end{verbatim}

You may need to make a change to the {\em sample.in} namelist.  
Open the {\em namelists.list} file and change the name of the first namelist to be consistent with the rest. For example:
(you can avoid this step by naming the BASE\_NAMELIST something like gmit\_sample\_1994\_jan.in)

Before:

\begin{verbatim}
        sample.in
        gmit_sample_1994_feb.in
        gmit_sample_1994_mar.in
        gmit_sample_1994_apr.in
        gmit_sample_1994_may.in
        gmit_sample_1994_jun.in
        gmit_sample_1994_jul.in 
        gmit_sample_1994_aug.in
        gmit_sample_1994_sep.in
        gmit_sample_1994_oct.in
        gmit_sample_1994_nov.in
        gmit_sample_1994_dec.in
\end{verbatim}

After:

\begin{verbatim}
        gmit_sample_1994_jan.in
        gmit_sample_1994_feb.in
        gmit_sample_1994_mar.in 
        gmit_sample_1994_apr.in
        gmit_sample_1994_may.in
        gmit_sample_1994_jun.in
        gmit_sample_1994_jul.in
        gmit_sample_1994_aug.in
        gmit_sample_1994_sep.in 
        gmit_sample_1994_oct.in
        gmit_sample_1994_nov.in
        gmit_sample_1994_dec.in
\end{verbatim}


Rename the file that you just changed. For example:

\begin{verbatim}
        %mv sample.in gmit_sample_1994_jan.in
\end{verbatim}

\begin{remark}
Make sure you have ALL your metFields files in the run directory, as the {\em ChainRuns.py}
script assumes this location.
\end{remark}

Modify your crontab to run the {\em ChainRuns.py} script at the proper time. 
For additional documentation on the crontab tool please reference:

\begin{verbatim}
        http://www.adminschoice.com/docs/crontab.htm
\end{verbatim}

Here is an example of a ChainRuns crontab entry:

\begin{verbatim}
        20 11 15 6 * /usr/bin/python /yourPath/ChainRuns/ChainRuns.py -d runDirectory >> /yourPath/Logs/fromCron.out
\end{verbatim}

In this example, the {\em ChainRuns.py} script will run at 11:20 am on June 15th.

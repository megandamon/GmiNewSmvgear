\chapter[Making Changes]{Making Changes to the Code} \label{chap:changes}
%
This chapter briefly describes some factors users should take into account
if they want to modify the code.

\section{Coding Consideration}
%
Some coding conventions must be followed when making code changes:
\begin{itemize}
\item All real variables must be declared "real*8".
\item All real numbers must have a "d" exponent even if "d0".
\item Must use "\# include" for all include statements.
\item Use only generic intrinsic function calls (for instance use MOD instead of AMOD).
\item Do not use the real or float intrinsic functions: just assign integer variable to a
      real*8 variable if need be and then proceed.
\item Avoid using machine-specific calls.
\item Use of appropriate F90 language features is encouraged (dynamic allocation, array
      syntax, etc.).
\item Use the physical constants and conversion factors defined in \\
      Shared/GmiInclude/gmi\_phys\_constants.h.
\item Use the time constants and conversion factors defined in \\
      Shared/GmiInclude/gmi\_time\_constants.h.
\end{itemize}

\section{Making Changes in the Code} \label{sec:changes}
%
Before making any change in the code, it is important to understand
its directory structure and the concept of components.
The code has been reorganized and major components were isolated
with the goals of making the code more readable, flexible,
accessible, modular and easy to maintain.
We grouped variables belonging to a given component into a derived
type.
The following derived types were created to classify and manipulate GMI variables:
%
\begin{description}
\item[Advection:] Contains advection related variables
\item[Chemistry:] Contains chemistry related variables
\item[Convection:] Contains convection related variables
\item[Deposition:] Contains deposition related variables
\item[Diffusion:] Contains diffusion related variables
\item[Emission:] Contains emission related variables
\item[SpeciesConcentration:] Responsible for manipulating species concentration related variables.
\end{description}
%
For each member variable (kept private and only directly accessible by the
component it belongs to) of the derived type, we wrote routines (whenever necessary)
to manipulate it: {\em Allocate} (to allocate the variable once), {\em Set} 
(to set the value of the variable anywhere in the code), and {\em Get} 
(to obtain the value of the variable anywhere in the code).
It is important to note that the three routines are the only operations directly done
on derived type member variables.

For each component, we wrote three new interface routines to standardize the way
components are handled:
%
\begin{description}
\item[InitializeComponent:] called once to read resource file (only the section
     corresponding to the component), do initial settings
     and perform variable allocation.
\item[RunComponent:] used to invoke the component for model time stepping.
\item[FinalizeComponent:] called if necessary to deallocate variables.
\end{description}
%
The above routines are the only ones accessible by the main program and
they completely hide the legacy system.
They have as arguments only derived types.
The only way to have access to a member variable of a derived type is through the
associated {\em Set} and {\em Get} routines.

In addition to the components listed above, we created three other components
to provide services to the seven main components.
They are:
\begin{description}
\item[gmiClock:] Contains clock information necessary for the model to advance in time.
\item[gmiGrid:] Contains grid-related data.
\item[gmiDomain:] Contains subdomain information.
\item[MetFields:] Responsible for updating all the meteorological related variables
     (by reading from a file or deriving from existing variables), and passing them
     to other components as needed.
\item[Diagnostics:] Responsible for setting the necessary diagnostics flags and
     passing the appropriate information to other components (that will do
     proper allocation of diagnostics variables they own).
\end{description}
%
\begin{example}
We want to provide an example on how the componentization was used to drive
the Chemistry component.
We list the arguments of the three routine interface:
\begin{verbatim}
    InitializeChemistry(Chemistry, Diagnostics, gmiConfig, gmiGrid, gmiDomain)
    RunChemistry       (Chemistry, Emission, MetFields, Diagnostics,
                        SpeciesConcentration, gmiClock, gmiGrid, gmiDomain)
    FinalizeChemistry  (Chemistry)
\end{verbatim}
Here, {\tt gmiConfig} is the the ESMF configure user file associated with the resource file.
\end{example}
%

\vskip 0.6cm

\noindent
The code associated with the first set of derived types is located in the 
{\em Components/} directory whereas the one for the second set is in the
{\em Shared/} directory (providing support to major components).
When making changes in the code, we need to identify where the modifications
should occur.
Assume that we want to add a new method for computing the tropopause pressure.
The new routine should be added in the directory {\em Shared/GmiMetFields/}
containing modules/routines manipulating metFields variables.

\subsection{Adding a Variable to a Derived Type}
%
Each derived type defined in the code is part of a module.
To add a new variable to a derived type, we need to do the following
operations:
\begin{enumerate}
\item Define the variable
\item Write a routine to allocate it (if an array)
\item Write a routine to get its value (if needed outside the component the derived type belongs to)
\item Write a routine to set its value (if updated outside the component the derived type belongs to)
\item Deallocate the derive type (if an array)
\end{enumerate}
%
As an example, assume that we want to output the overhead ozone column that is calculated 
inside the photolysis package.
The variable {\tt overheadO3col} needs to be created.
Since it will only be updated in the Chemistry component (where the Photolysis is located),
the variable will be part of the Chemistry component and should be a member of the Chemistry
derived type.
In addition, {\tt overheadO3col} will be visible outside the component.
The following will be added in the file:
%
\begin{verbatim}
    gmi_gsfc/Components/GmiChemistry/chemistryMethod/GmiChemistryMethod_mod.F90
\end{verbatim}
%
\begin{verbatim}
  real*8, pointer     :: overheadO3col(:,:,:)   => null()

  subroutine Allocate_overheadO3col (self, i1, i2, ju1, j2, k1, k2)
    integer           , intent(in   )  :: i1, i2, ju1, j2, k1, k2
    type (t_Chemistry), intent(inout)  :: self
    Allocate(self%overheadO3col(i1:i2, ju1:j2, k1:k2))
    self%overheadO3col = 0.0d0
    return
  end subroutine Allocate_overheadO3col

  subroutine Set_overheadO3col (self, overheadO3col)
    real*8          , intent(in)  :: overheadO3col(:,:,:)
    type (t_Chemistry), intent(inout)   :: self
    self%overheadO3col(:,:,:) = overheadO3col(:,:,:)
    return
  end subroutine Set_overheadO3col

  subroutine Get_overheadO3col (self, overheadO3col)
    real*8          , intent(out)  :: overheadO3col(:,:,:)
    type (t_Chemistry), intent(in)   :: self
    overheadO3col(:,:,:) = self%overheadO3col(:,:,:)
    return
  end subroutine Get_overheadO3col
\end{verbatim}

\subsection{Adding Chemical Mechanisms}
%
Currently, the GMI code has six chemical mechanisms:
\begin{enumerate}
\item {\em aerosol}
\item {\em micro\_aerosol}
\item {\em gocart\_aerosol}
\item {\em stratosphere}
\item {\em strat\_trop} (combined stratosphere/troposphere)
\item {\em troposphere}
\end{enumerate}
%
The code portion for each mechanism is contained in its own subdirectory 
(located at {\em Components/GmiChemistry/mechanisms/}) having the name of the corresponding 
mechanism.
It is organized into two subdirectories (see Chapter \ref{chap:structure}):
{\em include\_setkin/} and {\em setkin/}. 

If you want to add another chemical mechanism, just create a new subdirectory
from {\em Components/GmiChemistry/mechanisms/} and move the setkin files there.
To compile the code, follow the compilation procedures as described in Section
\ref{sec:compilation}.

The selection of a particular chemical mechanism is done through the 
environment variable {\em CHEMCASE} in the file {\em cshrc.gmi} 
(see Chapter \ref{chap:installation}).

\begin{remark}
It is important to note that if you make changes specific to a particular
chemical mechanism (i.e., outside setkin files), use the variable 
``chem\_mecha'' to delimit your changes.
\end{remark}

\subsection{netCDF Output Files}
We modified the process of producing netCDF output files.
All the operations needed to manipulate a file are now included in a unique
Fortran module.
In addition, the interface routines (initialize, control and finalize)
have standard interfaces and should not be changed.
If we want to add another diagnostics variable to a netCDF output file,
we may only have to make changes in the module: declare the variable,
allocate it, update its value every time step, communicate its value to
the root processor, write it out by the root processor and deallocate 
it.

%
%\section{The Make System}
%%
%\begin{description}
%\item[Makefile]: a template for Makefile.
%%\item[mkmf]: a command line that uses Makefile.cpp and the system files to
%%             produce the Makefile.
%%\item[make]: to compile the code (use regular make, not GNU make).
%%\item[make link]: for linking.
%\item[gmake clean]: to remove all the object files created at compilation.
%\item[gmake distclean]: to remove all temporary files and all the object files created at compilation.
%\end{description}
%
%%
%\section{Debugging the Code}
%If you want to run the code in a debug mode,
%\begin{itemize}
%\item Edit the file {\em Config/compiler.mk} and set the compilation options to "-g".
%\item Edit the file {\em Config/gem\_option.h} and set the parameter 
%      {\em Debug\_Option} to 1 and the other two options (optimization
%      and profiling) to 0.
%\item Go to the main directory
%\item Type: {\tt gmake all} 
%\item Submit the executable using a debugger such as TotalView
%%\end{itemize}
%

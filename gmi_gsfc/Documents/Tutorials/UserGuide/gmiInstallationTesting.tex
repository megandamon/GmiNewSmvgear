%
\chapter[Installation and Testing]{Installation and Testing}
\label{chap:installation}
%
This chapter is written to help new users install and 
test the GMI code. 
We provide specific instructions on how to obtain the code, 
to properly set environment variables, to select the model 
configuration, to choose a particular platform, 
to compile the code and to perform basic test runs. 
The focus of these instruction is on the installation and execution of 
the GMI code on {\tt discover} and {\tt explore}. 
The same procedures can easily be applied to any platform.

To get and install the GMI code, the following system software is needed:
%
\begin{itemize}
\item CVS (see Chapter \ref{chap:cvs} for instruction)
\item F90/95 (ideally {\em ifort} for intel)
\item C (ideally {\em icc} for intel). Down the road, it will not be required.
\item MPI 
\item netCDF (version 3.4 or higher). The location of netCDF should be
      provided in the files {\em Config/gem\_config.h} and {\em Config/compiler.mk}
      before compiling the code (see Section \ref{sec:envi} for details).
\item ESMF (see {\tt http://www.esmf.ucar.edu}) 
\item make and gmake
\item makedepend (generally in /usr/bin/X11)
\item perl
\item a debugger (if possible)
\end{itemize}


%
%\section{Installation and Testing}
During this process of installing and testing the code, it is 
assumed that {\em Cshell} is the default shell employed by the user.
In fact, the GMI environment variables required for these procedures are
set up using {\em Cshell}.

\section{Getting the Code}
%
To obtain the GMI code,
%
\begin{itemize}
\item Select the directory where you want to install the GMI model, 
      say {\em MYGMI/}
\item Get the latest version of the model from the cvs repository at 
      {\em sourcemotel} by typing the command lines:
\end{itemize}
%
\begin{verbatim}
        %setenv CVS_RSH ssh
        %cvs -d usrid@sourcemotel.gsfc.nasa.gov:/cvsroot/gmi co -P gmi_gsfc
\end{verbatim}
%
Here {\em usrid} is your login name on {\tt sourcemotel}.
You will be asked to provide your password on {\tt sourcemotel}. 
The directory {\em gmi\_gsfc/}, which is the main GMI directory,
is then created.
%
\section{Model Files and Directory Structure}
%
\noindent
Move into {\em gmi\_gsfc/}, the top directory of the GMI code:
%
\begin{verbatim}
        %cd gmi_gsfc
        %ls
\end{verbatim}
%
You will find (in {\em gmi\_gsfc/}) the files and directories:
%
\begin{verbatim}
    Applications/     Config/        README.first       Shared/       login.gmi
    CVS/              Documents/     README.install     cshrc.gmi
    Components/       Makefile       README.notice      gem/
\end{verbatim}
%
%The top directory of the GMI code is {\em gmi\_gsfc/} which contains
%sub-directories presented in Chapter \ref{chap:structure}.

%
\section{Setting Environment Variables} \label{sec:envi}
%
In the directory {\em gmi\_gsfc/}, read all the README files by 
starting with {\em README.first} 
file that guides a new user to take the required steps for 
installing and running the GMI code. 
The top portions of the files {\em cshrc.gmi} and {\em login.gmi}
(also located in {\em Shared/GmiScripts/})
include instructions for setting up the environment variables
which are discussed in this section.
\newline
\newline
%
Edit the file {\em cshrc.gmi} 
\begin{itemize}
\item Select the chemistry mechanism you want to consider by setting
      the variable {\em CHEMCASE}. Currently six mechanisms are
      available: {\em troposphere}, {\em aerosol}, {\em micro\_aerosol},
      {\em gocart\_aerosol}, {\em stratosphere}, and
      {\em strat\_trop} (for the combined stratosphere/troposphere or combo mechanism).
      If you want to use the {\em troposphere}  mechanism for instance, uncomment
      the corresponding line to have
\begin{verbatim}
        setenv CHEMCASE trop
\end{verbatim}
%
\item For the platform you want the GMI model to run on,
      update the variables {\em GMIHOME} (location of the main 
      model directory) and {\em GMI\_DATA} (directory where the input 
      data to test the installation are located - not really necessary):
%
\begin{verbatim}
        setenv GEMHOME   ~/MYGMI/gmi_gsfc
        setenv GMI_DATA  ~/MYGMI/gmi_gsfc
\end{verbatim}
\end{itemize}
%
%
Copy the files {\em cshrc.gmi} and {\em login.gmi}  from this directory to your home 
directory
%
\begin{verbatim}
        %cp cshrc.gmi ~/.cshrc.gmi
        %cp login.gmi ~/.login.gmi
\end{verbatim}
%
Go to the directory {\em Config/} and edit the file {\em gem\_sys\_options.h}. 
Modify the line
%
\begin{verbatim}
        #define ARCH_OPTION  ARCH_XXXX
\end{verbatim}
%
to select the architecture on which you want to run the code.
For instance, {\em XXXX} is {\em INTEL} for {\tt discover} or {\tt explore}. 
In this case, you also need to set
%
\begin{verbatim}
        #define HOST_MACH  YYYYY
\end{verbatim}
%
where {\em YYYYY} is either {\em DISCOVER} or {\em PALM} (for {\em explore}).
\newline
In addition, set the variable {\em MSG\_OPTION} to determine if you want a
single processor version of the code
%
\begin{verbatim}
        #define MSG_OPTION    MSG_NONE
\end{verbatim}
%
or a multiple processor version (using MPI) of the code
%
\begin{verbatim}
        #define MSG_OPTION    MSG_MPI
\end{verbatim}

You may also choose to edit the file {\em gem\_options.h} to select 
debugging, optimization, or profiling options. 
Provide the paths to MPI and netCDF include 
files and libraries in the files {\em gem\_config.h} and {\em libraries.mk}.  
Some compilation options may have to be changed in the files
{\em gem\_config.h} and {\em compiler.mk}.
%
Go to your home directory and edit the file {\em .cshrc}
%
\begin{verbatim}
        %cd ~/
        %vi .cshrc
\end{verbatim}
%
Include the lines
%
\begin{verbatim}
        setenv CVS_RSH ssh
        setenv ARCHITECTURE  ARCH_XXXX
        if (-e ~/.cshrc.gmi) then
           source ~/.cshrc.gmi
        endif
\end{verbatim}
%
You must also edit the {\em .login} file and add the lines
%
\begin{verbatim}
        if (-e ~/.login.gmi) then
            source ~/.login.gmi
        endif
\end{verbatim}
%
Update the changes made in the files {\em .cshrc} and {\em .login} by typing:
%
\begin{verbatim}
        %source .cshrc
        %source .login
\end{verbatim}
%

The setting of the environment variables ended with the previous two commands.
The setting automatically creates aliases that allow the user to easily access
the code directories and to execute scripts (see Chapter \ref{chap:script}).
For instance, by typing:
%
\begin{itemize}
\item {\tt cd gmi} or {\tt cd \$GMIHOME}, you will get to the code main directory.
\item {\tt cd phot}, you will move to the directory containing the Photolysis
      component package ({\em gmi\_gsfc/Components/GmiChemistry/photolysis/}).
\item {\tt cd emiss}, you will move to the directory containing the Emission
      component package ({\em gmi\_gsfc/Components/GmiEmission/}).
\item {\tt seabf my\_words}, you will search through all the GMI ".F90" files
      for the string {\em my\_words}.
\end{itemize}
%
%
\section{Code Compilation and Basic Test Run} \label{sec:compilation}
%
Go back to the working directory {\em MYGMI/gmi\_gsfc/}:
%
\begin{verbatim}
        %cd gmi
\end{verbatim}
%
\subsection{Compiling the model}
%

To compile the code, use the commands:
%
\begin{verbatim}
        %gmake all
\end{verbatim}
%
The {\em gmake} command compiles and links the code.
".f90", ".o", ".mod" and ".a" files are created and the executable named 
{\em gmi.x} is placed in the directory {\em Applications/GmiBin/}.

\subsection{Testing the Executable}

\noindent
To test the executable, we will use a sample resource file that comes with
the code.
For each platform, we show examples of job script files 
(named {\em gmitest.job}) to test the executable.
On {\tt explore} and {\tt discover} you need to have your
sponsor code account (type the command {\tt getsponsor} to obtain it).

It is assumed that the user wants to test the model from the directory
{\em /explore/nobackup/usrid} on {\tt explore} and
{\em /discover/nobackup/usrid} on {\tt discover}.

\begin{verbatim}
  runGmi_ExploreTest
        #PBS -S /bin/csh
        #PBS -N gmiCombo
        ##     -N sets job's name
        #PBS -l ncpus=64
        #PBS -l walltime=00:35:00
        #PBS -A a930b
        ##      -A sets the sponsor code account
        #PBS -V
        #PBS -e gmiCombo.err
        #PBS -o gmiCombo.out
        #
        setenv workDir /explore/nobackup/usrid
        setenv CHEMCASE strat_trop
        setenv GmiBinDir ~/MYGMI/gmi_gsfc/Applications/GmiBin

        cd $workDir
        #
        mpirun -np 64 $GmiBinDir/gmi.x
\end{verbatim}

\begin{verbatim}
  runGmi_DiscoverTest
        #PBS -S /bin/csh
        #PBS -N gmiCombo
        ###    -N sets job's name
        #PBS -l select=16:ncpus=4
        ###     choose 16 nodes and 4 processors per node
        #PBS -l walltime=00:35:00
        #PBS -W a930b
        ###     -W sets the sponsor code account
        #PBS -V
        #PBS -e gmiCombo.err
        #PBS -o gmiCombo.out
        #
        setenv workDir /discover/nobackup/usrid
        setenv CHEMCASE strat_trop
        setenv GmiBinDir ~/MYGMI/gmi_gsfc/Applications/GmiBin

        cd $workDir
        #
        limit stacksize unlimited
        mpirun -np 64 $GmiBinDir/gmi.x
\end{verbatim}

\begin{remark}
Replace a930b in the above script files with your sponsor code account.
Here {\em usrid} is the user's login name.
\end{remark}

To submit the job script, do the following
%
\begin{verbatim}
  On explore and discover
      %qsub runGmi_ExploreTest
      %qsub runGmi_DiscoverTest
\end{verbatim}
%
%
%\begin{verbatim}
%diff dao2rn_dao46_06.asc \
%     $GEMHOME/actm/gmimod/Other/test/par/outfiles/dao2rn_dao46_06.asc
%\end{verbatim}
%%
%The matching results ensure the installation and compilation of GMI 
%model is complete.
%
%\begin{remark}
%Due to the changes made within the code, it may turn out that the 
%verification process fails at the end of the run. 
%That does not mean the installation of the code was unsuccessful. 
%A failure may be due to the fact that the latest version of the code 
%contains several updates (as a result of the bugs that were found). 
%The comparison is done between the output from the new code and the 
%output of the original one (containing bugs).
%\end{remark}

\section{Summary of the Necessary Steps}
%
In this section, we summarize the steps needed to obtain, install,
and run the GMI code on any platform.
%
\begin{enumerate}
\item Obtain the code ({\em gmi\_gsfc} release) from the cvs repository.
\item Move to the GMI working directory ({\em gmi\_gsfc/}).
\item Edit the file {\em cshrc.gmi} to update the variables 
      {\tt GMIHOME}, {\tt GMI\_DATA} (not necessary) and {\tt CHEMCASE}.
\item Copy the files {\em cshrc.gmi} and {\em login.gmi} to {\em .cshrc.gmi} and 
      {\em .login.gmi} in your home directory.
\item Go to the directory {\em Config/} to edit the files
      {\em gem\_sys\_options.h}, {\em gem\_config.h}, {\em compiler.mk}, and
      {\em libraries.mk} to select the
      architecture and to update the compilation options and paths.
\item Go to your home directory to edit and source the files
      {\em .cshrc} and {\em .login}.
\item Type {\tt cd gmi} and compile the code by typing
      {\tt gmake all}.
\item Write a job script file and submit the job to run the executable.
\end{enumerate}

%%%%%%%%%%%%%%%%%%%%%%%%%%%%%%%%%%%%%%%%%%%%%%%%%%%%%%%%%%%%%%%%%%%%%%%%%%%%%%%
%   Abstracts                                                                 %
%%%%%%%%%%%%%%%%%%%%%%%%%%%%%%%%%%%%%%%%%%%%%%%%%%%%%%%%%%%%%%%%%%%%%%%%%%%%%%%
\chapter*{Abstract\markboth{Abstract}{Abstract}}
% Mark 'Abstract' both even and odd markers

In this report, we provide a description of the GMI code.
It is intended to help users in the GMI community
to obtain, install, compile, run, and modify the code.
We present the code organization and data structures,
procedures on how to run the code under various configurations,
to manipulate the code, and the code parallel
performance.  
Examples introduced here were carried out on Intel clusters.

\vskip 0.5cm

\noindent
This version of the User's Guide differs from the previous one
(released in 2004) in many areas due to the fact that the GMI
code went through a lot of changes. For instance:
%
\begin{itemize}
\item A completely new code directory structure
\item Componentization of the code
\item New resource file settings
\item New diagnostics capabilities.
\item Removal of the LLNL ESM package
\item Introduction of ESMF.
\end{itemize}
%
As a result of these changes, we have new installation and compilation
procedures, an object-oriented approach to manipulate components of
the code, a more flexible way to produce netCDF output file, a more
readable code, the ability to automatically generate code documentation, 
etc.

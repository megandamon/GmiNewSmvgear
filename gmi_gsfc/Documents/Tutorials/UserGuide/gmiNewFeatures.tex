\chapter[Important Features]{Important Features} \label{chap:features}

\section{From Species Indices to Species Names}
%
In older versions of the code,
setting resource file variables using species indices was not a simple task.
When moving from one experiment to another, users were be interested in
a given set of species but had to reset resource file variables matching
species with their indices.
In fact, their main concern was to know if a given species is part of
a chemical mechanism, but not its index.
Mistakes were unintentionally introduced, leading to a waste of time and
computing resources.
To alleviate these problems, we now use species labels instead
of indices in the resource file.

We wrote a Fortran module ({\em GmiSpeciesRegistry\_mod}
containing the function {\tt getSpeciesIndex}) providing the species index given
its name.
The function {\tt getSpeciesIndex} only takes a species name as its argument
and does not depend on any chemical mechanism or any particular experiment.
Two variables have to be passed to the module (will do internal initialization)
for the function to work properly:
%
\begin{itemize}
\item {\em num\_species}: total number of species used in the experiment
\item {\em const\_labels}: list of all species names used in the experiment.
\end{itemize}
%
\begin{remark}
The module gets its information at run time but not at compilation time.
This allows us to take into account tracer runs without making any specific
provision in the code.
\end{remark}

We substituted old namelist variables (using species indices) with
new resource file ones. In the resource file, we only need to provide the list of
species names we are interested in and the code will figure out
(using {\tt getSpeciesIndex}) what indices they correspond to.

In Table \ref{tab:nlNewVars}, we provide the list of the new resource file
variables and the corresponding old ones.
It is important to note that the old resource file variables remain part
of the code (as regular variables) and  are still used for calculations.
The newly created variables are only local variables utilized to
set the ones they replace in the resource file.

\begin{center}
\begin{longtable}{|l|l|} \hline\hline
{\bf Old Namelist variables} & {\bf New Resource File Variables} \\ \hline\hline
\multicolumn{2}{|c|}{\bf SpeciesConcentration SECTION } \\ \hline\hline
fixed\_const\_map(1:n)                      & fixedConcentrationSpeciesNames \\ \hline\hline
\multicolumn{2}{|c|}{\bf Diagnostics SECTION } \\ \hline\hline
pr\_const\_rec\_flag(1:n)                   & concentrationSpeciesNames   \\
pr\_emiss\_rec\_flag(1:n)                   & surfEmissSpeciesNames       \\
pr\_drydep\_rec\_flag(1:n)                  & dryDepSpeciesNames          \\
pr\_wetdep\_rec\_flag(1:n)                  & wetDepSpeciesNames          \\
pr\_tend\_rec\_flag(1:n)                    & tendSpeciesNames            \\
flux\_species(1:n)                          & fluxSpeciesNames            \\
ifreq$\#$\_species(1:n) ($\# = 1, 2, 3, 4$) & freq$\#$SpeciesNames        \\
species\_overpass$\#$(1:n) ($\# = 1, 2$)    & overpass$\#$SpeciesNames    \\
noon\_species(1:n)                          & noonSpeciesNames            \\
local\_species(1:n)                         & localSpeciesNames           \\
col\_diag\_species(1:n)                     & colDiagSpeciesNames        \\ \hline\hline
\multicolumn{2}{|c|}{\bf Advection SECTION } \\ \hline\hline
advec\_flag(1:n)                            & advectedSpeciesNames  \\ \hline\hline
\multicolumn{2}{|c|}{\bf Emission SECTION } \\ \hline\hline
emiss\_map(1:n)                             & emissionSpeciesNames  \\
emiss\_map\_aero(1:n)                       & emissionAeroSpeciesNames  \\
emiss\_map\_dust(1:n)                       & emissionDustSpeciesNames  \\ \hline\hline
\multicolumn{2}{|c|}{\bf Chemistry SECTION } \\ \hline\hline
forc\_bc\_map(1:n)                          & forcedBcSpeciesNames \\ \hline\hline
\caption{New resource file variables (and corresponding old namelist ones) used to set species names 
instead of species indices.}
\label{tab:nlNewVars}
\end{longtable}
\end{center}

We adopted the following principles for the new resource variables:
\begin{enumerate}
\item For a given variable, the number of species names entered is no longer necessary.
\item Each variable is a table with one entry per line.
\item In the previous version of the code, {\tt emiss\_map} was set in the
      namelist file to determine the number of species in the emission input
      file and to select the species to be read in from the file.
      In the new setting, if a species is not read in, the name to be
      included in {\tt emissionSpeciesNames} is xxx. The function will
      return -1 for the species index.
\item The order for entering the names is not important for diagnostics-related 
      variables. However it is relevant for variables
      ({\tt fixedConcentratinSpeciesNames, emission\-Species\-Names,
      emissionDustSpeciesNames, emissionAeroSpeciesNames,
      forced\-Bc\-Species\-Names}) used to read in files.
\item Entering species names is not case sensitive. For example, the names
     {\tt HNO3, hNO3, HnO3, HNo3, hnO3, hNo3, Hno3, hno3} correspond to the
     same species. Users can select any of these names to refer to {\tt HNO3}.
\item If a species names does not exist, the code will abort.
\end{enumerate}

\begin{remark}
In previous versions of the GMI code, it was assumed that if a set of
species is selected for constituent, wet deposition, dry deposition,
and tendency output, the first species in the mechanism will be included by
default. In this work, we did not make such an arrangement.
Only the species provided by the user are considered for output.
\end{remark}

Here is an example of resource file setting for the AURA (combo, 124 species, no ship emission)
 experiments:

{\small
\begin{verbatim}
  dryDepSpeciesNames::
  CH2O
  H2O2
  HNO3
  MP
  N2O5
  NO2
  O3
  PAN
  PMN
  PPN
  R4N2
  ::

  forcedBcSpeciesNames::
  CFCl3
  CF2Cl2
  CFC113
  CFC114
  CFC115
  CCl4
  CH3CCl3
  HCFC22
  HCFC141b
  HCFC142b
  CF2ClBr
  CF2Br2
  CF3Br
  H2402
  CH3Br
  CH3Cl
  CH4
  N2O
  ::
\end{verbatim}
}

%
\section{Station Diagnostics}
In previous versions of the code, when we wanted to do station diagnostics we
needed to set in the namelist file three variables :
%
\begin{description}
\item[col\_diag\_num:] total number of stations
\item[col\_diag\_site:] complete list of stations
\item[col\_diag\_lat\_lon:] locations (latitute/longitude) of stations.
\end{description}
%
Users were required not only to count the number of stations (can be several hundreds)
 but also to match the name of each station with its location.
Removing/adding one station from the list involved the resetting of
the above three variables with the possibility of mismatching.
To facilitate the selection of stations, we :
%
\begin{itemize}
\item Constructed a file containing a list of all the identified stations and their locations.
      A resource file variable was created to point to the file. New stations can be added
      to the file at any time. The order of writting station information is irrelevant.
\item Added a new resource file variable (long string) to enter the list of selected stations.
\item Made changes in the code to check if each selected station exists in the file and then extract
      its location.
\item Computed the number of selected stations at run time.
\end{itemize}
%
We created two new resource file variables (see Table \ref{tab:nlColDiag}).
%
\begin{center}
\begin{longtable}{|l|l|} \hline\hline
{\bf Old Namelist variables} & {\bf New Resource File Variables} \\ \hline\hline
\multicolumn{2}{|c|}{\bf Diagnostics SECTION } \\ \hline\hline
col\_diag\_num            & N/A     \\
col\_diag\_site()         & colDiagStationsNames \\
col\_diag\_lat\_lon(2,n)  & N/A                  \\
N/A                       & stationsInputFileName \\ \hline\hline
\caption{New resource file variables for station diagnostics.}
\label{tab:nlColDiag}
\end{longtable}
\end{center}
%
The piece of code used to carry out the above operations is:
%
\begin{verbatim}
      ! Construct the list of station using the long string
      call constructListNames(col_diag_site, colDiagStationsNames)

      col_diag_num = Count (col_diag_site(:) /= '')

      if (col_diag_num /= 0) then

         do ic = 1, col_diag_num
            ! For each station in the list, check if it exists in the file
            ! and get its position (lat/lon)
            call getStationPosition(col_diag_site(ic), &
                                    col_diag_lat_lon(1,ic), &
                                    col_diag_lat_lon(2,ic), &
                                    stationsInputFileName)
         end do
          .
          .
          .
      end if
\end{verbatim}
%
We present below a sample resource file setting for {\tt colDiagStationsNames}
and the first few lines of the file ({\tt colDiagStationList.asc})
 containing station information.

\begin{verbatim}
 colDiagStationsNames::
 SPO
 MCM
 HBA
 FOR
 NEU
 SYO
 PSA
 MAR
 MAQ
 WAT
 ASC
 NAT
 SEY
 BRA
 MAL
 NAI
 SNC
 CHR
 KCO
 PAR
 TVD
::

# --------------------------------------------------------------
# Station name       Lat     Lon                 Description
# --------------------------------------------------------------
SPO                -89.98   335.20
MCM                -77.83   166.60
HBA                -75.56   333.50
FOR                -71.00    12.00
NEU                -71.00   352.00
SYO                -69.00    39.58
PSA                -64.92   296.00
\end{verbatim}

\begin{remark}
While editing the file containing the station information, the following rules apply:
\begin{enumerate}
\item The first three lines of the file should start with the $\#$ character.
\item Column 1 to Column 16 are for the station name.
\item Column 17 to Column 25 are for the latitude of the station.
\item Column 26 to Column 34 are for the longitude of the station.
\item The remaining columns are reserved to describe stations and are not read in.
\end{enumerate}
\end{remark}

\section{Frequency Diagnostics}
With the Freq files, users can select the variables they desire
to output at any frequency:
\begin{enumerate}
\item {\em const} (only for selected species; written out if {\em freq$\#$SpeciesNames} is set)
\item {\em constSurface} (only for selected species; written out if {\em freq$\#$SpeciesNames} and {\em pr\_const\_surface\_freq$\#$} are set)
\item {\em constColTrop} (only for selected species; written out if {\em freq$\#$SpeciesNames}  and {\em pr\_const\_column\_freq$\#$} are selected  only when using troposphere or combo mechanism)
\item {\em constColCombo} (only for selected species; written out if {\em freq$\#$SpeciesNames}  and {\em pr\_const\_column\_freq$\#$} are selected only when using the combo mechanism)
\item {\em kel}       (written out if {\em pr\_kel\_freq$\#$}=T)
\item {\em psf}       (written out if {\em pr\_psf\_freq$\#$}=T)
\item {\em metwater} (written out if {\em pr\_metwater\_freq$\#$}=T)
\item {\em mass}       (written out if {\em pr\_mass\_freq$\#$}=T)
\item {\em relHumidity}      (written out if {\em pr\_rel\_hum\_freq$\#$}=T)
\item {\em overheadO3Column}   (written out if {\em pr\_overheadO3col\_freq$\#$}=T)
\item {\em tropopausePressure} (written out if {\em pr\_tropopausePress\_freq$\#$}=T)
\item {\em potentialVorticity} (written out if {\em pr\_potentialVorticity\_freq$\#$}=T)
\end{enumerate}
%
Here  $\#$ is 1, 2, 3 or 4.

There are other Freq related resource file variables that are of interest:
%
\begin{verbatim}
        freq#_description: short description of the file
        freq#_name       : name of the Freq file

         pr_nc_freq#     : frequency for the outputs

   Selected range of vertical levels
        k1_freq#         : lowest  level to output
        k2_freq#         : highest level to output

   Selected horizontal domain for output
        lonRange_freq#(1): west  longitude between 0 and 360
        lonRange_freq#(2): east  longitude between 0 and 360
        latRange_freq#(1): south latitude  between -90 and 90
        latRange_freq#(2): north latitude  between -90 and 90
\end{verbatim}


\section{Overpass Diagnostics} \label{sec:noon_variable}
Users  can select a range of time for which they wish to save out
selected variables and to write them in the file, $<$problem\_name$>$.overpass$\#$.nc 
(where $\#$ is 1, 2, 3 or 4).
The variables are:
\begin{enumerate}
\item {\em const} (only for selected species; written out if {\em overpass$\#$SpeciesNames} is set)
\item {\em kel}       (written out if {\em do\_mean}=T, and {\em pr\_kel\_overpass$\#$}=T)
\item {\em psf}       (written out if {\em do\_mean}=T, and {\em pr\_psf\_overpass$\#$}=T)
\item {\em metwater} (written out if {\em do\_mean}=T,  and {\em pr\_metwater\_overpass$\#$}=T)
\item {\em qj}       (written out if {\em do\_mean}=T, and {\em pr\_qj\_overpass$\#$}=T)
\item {\em qqj}      (written out if {\em do\_mean}=T, and {\em pr\_qqjk\_overpass$\#$}=T)
\item {\em qqk}      (written out if {\em do\_mean}=T, and {\em pr\_qqjk\_overpass$\#$}=T)
\end{enumerate}
%
Four new resource file variables were added
\begin{description}
\item[pr\_overpass$\#$]: if set to true, the overpass output file is produced. 
      The file will still be created if any of the above conditions (resource file setting)
      is satistfied.
\item[begTime\_overpass$\#$]: begin time. 
\item[endTime\_overpass$\#$]: end time.
\item[pr\_overpass$\#$\_period\_days]: overpass variable output period.
\end{description}
%
A sample resource file setting looks like (in the Diagnostics section):
%
\begin{verbatim}
  do_mean: T
  pr_overpass1: T
  overpass1SpeciesNames::
  CH2O
  CO
  NO
  NO2
  O3
  OH
  ::
  begTime_overpass1: 7.0d0
  begTime_overpass1: 16.0d0
  pr_overpass1_period_days: -1.0d0
\end{verbatim}

In the above resource file setting, {\em pr\_overpass1} = T. 
Therefore the file $<$problem\_name$>$.overpass1.nc will be created. 
It will contain information (between 7am and 4pm) of the variables 
{\em const\_overpass}, {\em kel\_overpass}, and {\em psf\_overpass} 
(the resource file variables {\em pr\_const\_overpass1}, {\em pr\_kel\_overpass1}, 
and {\em pr\_psf\_overpass1} will automatically be set to true). 
One can obtain the mean values of the variables {\em metwater\_overpass}, 
{\em qj\_overpass}, {\em qqj\_overpass} and {\em qqk\_overpass} by setting the resource file 
variables {\em pr\_metwater\_overpass1}, {\em pr\_qj\_overpass1},
{\em pr\_qqj\_overpass1}, and {\em pr\_qqk\_overpass1}, respectively, to true.
It is important to note the following:
%
\begin{enumerate}
\item We have assumed that {\em do\_mean}=T in order to be able to produce the 
       overpass variables.
\item If {\em pr\_overpass1} is not set at all in resource file, the code will
      automatically set it to true if {\em overpass1SpeciesNames} has at least one species.
\item If {\em pr\_overpass1} is set to false in the resource file, then
       the file $<$problem\_name$>$.overpass1.nc will not be created at all,
       regardless of the setting of the other variables.
\end{enumerate}

\section{Choice of Vertical Levels}
All 3D variables are currently saved out on all vertical levels by default.
However the user may only want to carry out post-processing
analyses over a specific range of levels.
At run time, the user can set the following resource file variables to select
the range of vertical levels to output (this does not apply to the freq$\#$,
overpass$\#$, and restart files):
%
\begin{description}
\item[pr\_level\_all]: if set to T, all the vertical levels are
   considered, otherwise a range (selected in the next two variables)
   is used.
\item[k1r\_gl]: First global altitude index for output.
   Should be at least equal to {\tt k1\_gl}.
\item[k2r\_gl]: Last global altitude index for output.
   Should be at most equal to {\tt k2\_gl}.
\end{description}
%
A sample resource file setting looks like (in the Control section):
\begin{verbatim}
   pr_level_all: F
   k1r_gl: 3
   k2r_gl: 20
\end{verbatim}
%
When {\tt k1r\_gl} and {\tt k2r\_gl} are set, they apply to all the netCDF output
files except the restart file.

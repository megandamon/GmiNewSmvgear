%
\chapter[Performing Specific Runs]{Performing Specific Runs}
\label{chap:runs}
%
%
\section{The GMI Self  Contained Test} \label{sec:SelfContainedTest}
%
For testing the model, we provide the directory {\em gmi\_gsfc/SelfContainedTest}.
We assume that the test is being run on {\tt discover} and uses the combo (strat\_trop)
mechanism.  In this section we provide intructions for running a 2x2.5
resolution combo model simulation.

The latest version of the code requires an appropriate namelist file.  We provide
this namelist file, {\em gmi\_gsfc/SelfContainedTest/namelist2x2.5.in}.
This namelist file does not require modification to run the test.  
All input files required to run the test are located in the
{\em gmi\_gsfc/SelfContainedTest/input-2x2.5} directory; the namelist uses relative
paths to locate the input files here.

Edit the file {\em gmi\_gsfc/SelfContainedTest/run\_ref} and provide the full location 
of the 
{\em gmi\_gsfc/SelfContainedTest} directory (no relative paths) by 
setting 
%
\begin{verbatim}
        setenv workDir /discover/nobackup/userid/gmi_gsfc/SelfContainedTest
\end{verbatim}
%
to the appropriate path.

The test can now be run by typing
%
\begin{verbatim}
        %qsub run_ref
\end{verbatim}
%

Once the test has finished the standard output can be checked for a succesful run.
The test run should have created one day's worth of output data.  The standard 
output file should include a statement such as
%
\begin{verbatim}
        --------  Successful completion of the run  -----------
\end{verbatim}

\section{Standard Runs} \label{sec:StandardRuns}

The two files mentioned in the preceding section, {\em gmi\_gsfc/SelfContainedTest/namelist2x2.5.in} and
{\em gmi\_gsfc/SelfContainedTest/run\_ref} can be used a basis for other types of
runs, like a troposphere run, for example.  To use the various model configurations, the code
will need to be compiled with one of the following options:
%
\begin{verbatim}
        setenv CHEMCASE troposphere
        setenv CHEMCASE stratosphere
        setenv CHEMCASE strat_trop
        setenv CHEMCASE strat_trop_aerosol
        setenv CHEMCASE aerosol
        setenv CHEMCASE micro_aerosol
        setenv CHEMCASE gocart_aerosol
\end{verbatim}
%

The {\tt $CHEMCASE$} option should also be changed in the {\em run\_ref} 
file.

\section{Other Namelist Variables} \label{sec:NamelistVariables}

Three important namelist variables are used to set the number of
processors.
They are {\tt $numWorkerProcs$}, {\tt $numLonProcs$} and {\tt $numLatProcs$}, and
they satisfy the relation 
%
$$numWorkerProcs = numLonProcs \times numLatProcs.$$
%
To execute the code, the number of processors {\tt $Ncpus$} is given
by {\tt $Ncpus = numWorkerProcs + 1$}.

The resolution of the run is set using the following variables: 
{\tt $i1\_gl$} and {\tt $i2\_gl$} (first and last index of the global longitude); 
{\tt $ju1\_gl$} and {\tt $jv1\_gl$} (first global `u` and `v` latitude); 
{\tt $j2\_gl$} (last global `u\&v` latitude); {\tt $kl\_gl$} and 
{\tt $k2\_gl$} (first and last global altitude).

\begin{remark}
The resolution of the run that is specified by the above variables should 
match the resolution of the metfield files.  
Metfield files can be found on {\tt discover} in
%
\begin{verbatim}
    /discover/nobackup/projects/gmi/gmidata2/input/metfields/TYPE/RESOLUTION/
\end{verbatim}
%

and on explore/palm in
%
\begin{verbatim}
     /explore/nobackup/projects/gmi/gmidata2/input/metfields/TYPE/RESOLUTION/
\end{verbatim}
%

The metfield files are listed in a file specified by the namelist variable 
{\tt $met\_filnam\_list$}.
\end{remark}


\section{Restart Capabilities} \label{sec:Restart}
To restart from a previous run, in the nlGmiRestart section of the input namelist file, set
%
\begin{verbatim}
     pr_restart: T
     pr_rst_period_days: #.#d0,
\end{verbatim}
%
   Replace $"\#.\#"$ with whatever you want the time period to be. \\
%
   Note that pr\_rst\_period\_days when converted to seconds, must be
   a multiple of mdt, the time increment for reading in new met data.
%
Restart netCDF output will be written to a file named
%
\begin{verbatim}
     <problem_name>.rst.nc
\end{verbatim}
%
   Set the resource file variable, {\em do\_overwrt\_rst} as appropriate
   (i.e., do you want to keep just a single record of restart data or
   multiple records?).

\vskip 1.0cm

\noindent
{\bf To create a new namelist input file to use for running from a
restart point:}
%
\begin{enumerate}
\item Section 7 provides information on a script called {\em MakeNameLists.py} that
can create multiple namelist files and handle restarting.
\item Alternatively, you can execute the {\em mk\_rstnl}
      ("make restart namelist") script as follows
\begin{verbatim}
     $gmi/Shared/GmiScripts/mk_rstnl \
        -onl <old_nlfile>   -nnl <new_nlfile>  \
        -rst <rst_ncfile>   -eda <endGmiDate>  -eti <endGmiTime> \
       [-npn <problem_name>]

   The first four arguments/value pairs are required

     old_nlfile   : name of the namelist input file that was used to get to
                    the restart point
     new_nlfile   : name to call the new namelist input file that will be
                    used to continue from the restart point
     rst_ncfile   : name of the netCDF restart file that was written out at
                    the restart point
     endGmiDate   : ending date for the new run; note that begGmiDate will 
                    be set by the script to what endGmiDate was when the 
                    netCDF restart file was written out
     endGmiTime   : ending time for the new run; note that begGmiTime will 
                    be set by the script to what endGmiTime was when the 
                    netCDF restart file was written out

   The last argument/value pair is optional

     problem_name : new problem name to use in the new namelist input file
\end{verbatim}
\end{enumerate}
%
For example
\begin{verbatim}
     $gmi/Shared/GmiScripts/mk_rstnl -onl nlfile.in.old  -nnl nlfile.in.new \
       -rst ncfile.rst.nc  -eda 20061231 -eti 240000
\end{verbatim}
%
This script automatically creates a new namelist file with various namelist
variables modified or added to accomodate the restart.  The script uses
 information from the netCDF restart file to accomplish this.  Values that
the script will use are output to the screen; these should be reviewed to
verify that they are what you expected them to be.
\newline
\newline
The namelist variables that the script will potentially modify or add are:
%
\begin{verbatim}
     nlGmiControl =>
       problem_name        : set to value provided by -npn script argument;
                             otherwise unchanged from value in old_nlfile
       begGmiDate          : set to endGmiDate at restart point
       begGmiTime          : set to endGmiTime at restart point
       endGmiDate          : set to value provided by -tfi script argument
       endGmiTime          : set to value provided by -tfi script argument
       gmi_sec             : set to ending value at restart point

     nlGmiMetFields   =>
       met_infile_num      : set to ending value at restart point
       mrnum_in            : set to ending value at restart point - 1
       tmet1               : set to tmet2        at restart point

     nlGmiDiagnostics  =>
       pr_qqjk             : set to ending value at restart point

     nlGmiRestart =>
       rd_restart          : set to T
       restart_infile_name : set to value provided by -rst script argument
\end{verbatim}
%
Note that restart dumps can only occur at the end of a met data cycle and
you should edit the new namelist input file directly if you want to change
any other namelist variables not listed above.

If there is more than one record in the restart file, the default is to
use the last one.  This can be changed by setting the namelist variable,
{\em restart\_inrec}, to a different record number.

\vskip 1.0cm

\noindent
{\bf To resume running from the restart point: }
%
\begin{itemize}
\item Start the run just as you normally would, but this time use the new
   restart namelist input file you created above.
\end{itemize}

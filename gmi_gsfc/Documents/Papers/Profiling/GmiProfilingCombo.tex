%\documentstyle[11pt,twoside,psfig]{article}
\documentstyle[11pt,twoside]{article}
%\input{psfig}
\pagestyle{plain}
\renewcommand{\baselinestretch}{1.0} % to make document single-spaced
%\textwidth17.5cm \textheight22.54cm
\textwidth15.5cm \textheight22.0cm
\oddsidemargin0mm \evensidemargin-4.5mm \topmargin0.50cm %-27mm
%\parskip10pt
\parindent 25pt

\begin{document}
%
\title{\Large\bf GMI Reproducibility}
\author{Jules Kouatchou \\
        AMTI - NASA/GSFC Code 931 \\
        Greenbelt, MD 20771}
\date{March 18, 2004}
\maketitle

\begin{abstract}
We carry out a set of experiments of analyze the bitwise reproducibility
of the GMI model on \textit{daley}. We want to verify if for a particular
model configuration the GMI code produces exactly the same output files
(1) when it is ran many times on the same number of processors
(weak reproducibility), or/and
(2) when it is ran on different number of processors
(strong reproducibility).
\end{abstract}

Here KBLOOP is the maximum number of grid-points in a vectorized block;
should range from  512 (below which vectorization decreases) to 1024
(above which array space is limited).
KBLOOP is used to determine the value of of KTLOOP that is the number
of grid-cells in a grid-block.
When KBLOOP=1 then KTLOOP=1 on any processor otherwise the value of
KTLOOP varies from one processor to another.
KBLOOP=75 seems to be the optimum value (or close to the optimum)
providing KTLOOP values producing a significant gain in speed with no
loss of accuracy on massively parallel computers \cite{Rotman-etal01}.

\begin{table}[!h]
\begin{center}
\begin{tabular}{|c|c|c|c|} \hline\hline
 Species & Thermal Reactions & Photolytic Reactions & Advected Species \\ \hline\hline
 124 & 320 & 81 & 69 \\ \hline\hline
\end{tabular}
\caption{Basic information on the chemical mechanism.}
\label{tab:mecha}
\end{center}
\end{table}

\begin{table}[!h]
\begin{center}
\begin{tabular}{|c|c|c|c|} \hline\hline
Configuration 1 & Configuration 2 & Configuration 3 & Configuration 4 \\ \hline\hline
75   &   50           &   90           &   25           \\ \hline\hline
\end{tabular}
\caption{Values of $KBLOOP$.}
\label{tab:config}
\end{center}
\end{table}


\begin{table}[!h]
\begin{center}
\begin{tabular}{||c||c|c|c||c|c|c||c|c|c||} \hline\hline
 & \multicolumn{9}{|c|}{\bf Processor Decomposition} \\ \hline
 & \multicolumn{3}{|c|}{$8 \times 8$} & \multicolumn{3}{|c|}{$16 \times 8$} & \multicolumn{3}{|c|}{$8 \times 16$} \\ \hline & min & max & avg & min & max & avg & min & max & avg \\ \hline\hline
Whole GMI           &862.71&862.72&862.71&509.12&509.21&509.15&497.86&497.86&497.86\\
Time Stepping       &787.57&789.92&788.68&431.39&434.51&432.47&423.21&426.83&425.30\\
Advection           &78.056&78.670&78.299&50.798&51.152&50.946&50.385&51.024&50.769\\
Chemistry           &135.71&517.77&345.73&68.650&260.23&178.63&59.016&284.53&171.13\\
procSyncEndStepping &145.86&532.22&320.21&97.103&288.11&177.62&63.767&292.64&178.60 \\
writingOutput       &32.264&34.183&33.113&27.226&29.263&27.795&25.908&27.855&26.703\\ \hline\hline
\end{tabular}   
\caption{Profiling for Configuration 1.}
\label{tab:wretime1}
\end{center}
\end{table}


\begin{table}[!h]
\begin{center}
\begin{tabular}{||c||c|c|c||c|c|c||c|c|c||} \hline\hline
 & \multicolumn{9}{|c|}{\bf Processor Decomposition} \\ \hline
 & \multicolumn{3}{|c|}{$8 \times 8$} & \multicolumn{3}{|c|}{$16 \times 8$} & \multicolumn{3}{|c|}{$8 \times 16$} \\ \hline & min & max & avg & min & max & avg & min & max & avg \\ \hline\hline
Whole GMI           &692.22&692.22&692.22&404.85&404.87&404.86&409.35&409.38&409.36\\
Time Stepping       &617.96&620.49&619.24&335.30&337.11&336.07&335.18&337.98&336.60\\
Advection           &77.743&78.392&78.001&51.196&51.574&51.362&50.300&50.875&50.687\\
Chemistry           &116.69&379.59&268.31&58.291&186.78&133.51&51.161&199.25&133.33\\
procSyncEndStepping &114.74&382.26&228.53&72.775&203.04&126.80&61.551&211.70&127.88 \\
writingOutput       &32.894&34.636&33.515&28.131&29.294&28.514&26.207&28.368&26.955\\ \hline\hline
\end{tabular}   
\caption{Profiling for Configuration 2.}
\label{tab:wretime2}
\end{center}
\end{table}


\begin{table}[!h]
\begin{center}
\begin{tabular}{||c||c|c|c||c|c|c||c|c|c||} \hline\hline
 & \multicolumn{9}{|c|}{\bf Processor Decomposition} \\ \hline
 & \multicolumn{3}{|c|}{$8 \times 8$} & \multicolumn{3}{|c|}{$16 \times 8$} & \multicolumn{3}{|c|}{$8 \times 16$} \\ \hline
 & min & max & avg & min & max & avg & min & max & avg \\ \hline\hline
Whole GMI           &958.39&958.41&958.39&559.18&559.19&559.18&553.12&553.17&553.13\\
Time Stepping       &888.31&890.35&889.19&489.63&491.13&490.38&482.44&486.18&483.91\\
Advection           &77.784&78.397&78.029&51.147&51.552&51.301&50.109&50.709&50.479\\
Chemistry           &147.34&605.69&392.67&75.132&315.61&206.59&63.764&336.9247&194.83\\
procSyncEndStepping &161.72&621.44&374.56&100.13&340.82&208.77&70.145&346.37&213.77\\
writingOutput       &32.448&34.056&33.041&28.292&29.293&28.671&25.270&27.416&26.140\\ \hline\hline
\end{tabular}
\caption{Profiling for Configuration 3.}
\label{tab:wretime3}
\end{center}
\end{table}

\begin{table}[!h]
\begin{center}
\begin{tabular}{||c||c|c|c||c|c|c||c|c|c||} \hline\hline
 & \multicolumn{9}{|c|}{\bf Processor Decomposition} \\ \hline
 & \multicolumn{3}{|c|}{$8 \times 8$} & \multicolumn{3}{|c|}{$16 \times 8$} & \multicolumn{3}{|c|}{$8 \times 16$} \\ \hline
 & min & max & avg & min & max & avg & min & max & avg \\ \hline\hline
Whole GMI           &636.84&636.85&636.84&394.99&395.00&394.99&389.96&390.03&389.99 \\
Time Stepping       &567.98&570.19&569.18&314.72&315.52&315.11&313.93&319.21&317.07\\
Advection           &77.395&77.887&77.622&51.008&51.349&51.129&53.186&53.785&53.56\\
Chemistry           &106.48&343.93&250.18&52.882&168.66&123.04&47.272&182.61&124.15\\
procSyncEndStepping &103.70&342.31&198.63&70.179&186.98&117.02&53.525&190.84&112.52\\
writingOutput       &32.277&34.238&32.837&29.054&30.395&29.388&16.163&29.993&26.947\\ \hline\hline
\end{tabular}
\caption{Profiling for Configuration 4.}
\label{tab:wretime4}
\end{center}
\end{table}

%*** The following command produces a extra blank line.
\begin{thebibliography}{99}

\bibitem{Drake-etal} J. Drake, I. Foster, J. Michalakes, B. Toonen and
    P. Worley,
    Design and performance of a scalable community climate model
    {\em Parallel Computing}, {\bf 21}, p. 1571- (1995).

\bibitem{Rotman-etal01} D.A. Rotman, et al,
    Global modeling initiative assessment model: model description,
    integration, and testing of the transport shell,
    {\em J. Geo. Research}, {\bf 106}(D2), p. 1669-1691 (2001).

\end{thebibliography}
\end{document}

